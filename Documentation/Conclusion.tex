\newpage
\part*{\large \centering ВЫВОДЫ}
\addcontentsline{toc}{part}{ВЫВОДЫ}
\begin{itemize}
\item Предложен и реализован новый метод парного выравнивания последовательностей ДНК с учетом сдвигов рамки считывания.
\item На его основе создана программа multy для решения задачи множественного выравнивания, которая позволяет получать ответ как для самих нуклеотидных последовательностей, так и для продуктов их трансляции.
\item Проведено сравнение multy с единственным существующим аналогом --- программой MACSE на демонстрационном наборе задач. Показано, что результаты множественного выравнивания согласуются с эталонными и что multy превосходит MACSE по производительности в 20,9 раз на наборе из 15 последовательностей длины 2500 нуклеотидов.
\end{itemize}   

\newpage
\part*{\large \centering ЗАКЛЮЧЕНИЕ}
\addcontentsline{toc}{part}{ЗАКЛЮЧЕНИЕ}
\hspace{\parindent}В настоящей работе предложен и реализован новый метод построения множественных выравниваний кодирующих последовательностей ДНК с учетом сдвигов рамки считывания. Отличительной особенностью представленного решения является то, что в отличие от классического алгоритма Нидлмана-Вунша, при построении парных выравниваний используются не только сами нуклеотидные последовательности, но также их возможные продукты трансляции. Такой подход позволяет не просто решать математическую задачу выравнивания строк алфавита генетического кода, но учитывать ее биологический контекст. Представленный в работе метод позволяет снизить влияние ошибок при секвенировании нуклеотидных последовательностей, <<преждевременных>> стоп-кодонов, вызванных ошибками или мутациями и других артефактов.\\
\indent Разработанный алгоритм построения множественных выравниваний реализован в виде кросс-платформенной компьютерной программы на языке C++, а также в виде набора статических библиотек, которые можно включать в состав сторонних программ. Важной особенностью представленной программы является ее модульность: функции каждого из этапов построения выравнивания реализованы независимо и могут быть заменены на другие. В частности, в текущей версии для построения матрицы сходства для набора последовательностей можно использовать либо метод попарных выравниваний, либо метод подсчета общих подстрок заданной длины. В плане дальнейшей разработки программы стоит добавление альтернатив алгоритму кластеризации UPGMA, а также ряда методов, позволяющих улучшить промежуточные результаты выравнивания.
