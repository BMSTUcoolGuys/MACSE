\newpage
\part*{\large \centering ВВЕДЕНИЕ}
\addcontentsline{toc}{part}{ВВЕДЕНИЕ}
\hspace{\parindent} Биоинформатика --- это молодая область науки возникшая в 1976-1978 годах и сформировавшаяся в 1980 году. В настоящее время она переживает бурный рост связанный с развитием техники получения и обработки биологической информации. По сути, это собрание различных математических моделей и методов в помощь биологам для решения биологических задач, таких как: предсказание пространственной структуры белков, расшифровка структуры ДНК, хранение, поиск и аннотация биологической информации.\\
\indent Одной из ключевых задач в биоинформатике является оценка сходства последовательностей. Ее решение позволяет, например, описывать эволюционный путь генетических элементов, предсказывать связь между структурой генетического материала и его функциональностью.\\
\indent Для того чтобы определить, насколько две биологические последовательности <<похожи>>, их представляют в виде строк символов из алфавита соответствующего нуклеотидам и аминокислотам. К ним применяют алгоритмы выравнивания, основанные на размещении исходных последовательностей мономеров ДНК, РНК или белков друг под другом таким образом, чтобы максимизировать некоторую функцию оценки результата~\cite{WikiPairAlign}. Качество выравнивания определяют назначая штрафы за несовпадение букв и за наличие пробелов (когда приходится разрывать одну последовательность для того, чтобы получить наибольшее число совпадающих позиций), например, через расстояние Левенштейна --- минимальное число элементарных операций (вставка, удаление или замена символа в строке), чтобы превратить одну строку в другую~\cite{Levenshtein}. При построении выравнивания решается задача максимизации выбранной функции оценки. В такой постановке рассматривается задача получения <<глобального выравнивания>>. Необходимо отметить, что для полных геномов глобальное выравнивание не работает, так как при мутации, помимо вставок, удалений и замен, бывают нелинейные перестройки, которые могут менять порядок и ориентацию целых геномных блоков. Для решения, аналогично задаче поиска глобального выравнивания, формулируют задачу поиска <<локального выравнивания>>: для двух произвольных строк $A$ и $B$ найти две самые похожие подстроки и их выравнивание.\\ 
\indent Алгоритмы множественного выравнивания, аналогично алгоритмам парного выравнивания, представляют собой инструмент для установления функциональных, структурных или эволюционных взаимосвязей между биологическими последовательностями.  Несмотря на то что задача множественного выравнивания была сформулирована более 20 лет назад~\cite{SIAM_Journal}, она до сих пор не теряет своей актуальности. Если говорить о множественном глобальном выравнивании, то, по сравнению с парным выравниванием, практически ничего не меняется: необходимо расставить разрывы в выравниваемых строках таким образом, чтобы <<счет по столбцам>> был максимален. Счет по столбцу можно вести, перебирая все пары символов. Множественное локальное выравнивание обобщить на многомерный случай не так просто. Во-первых, какие-то подстроки могут быть не во всех последовательностях. Во-вторых, последовательности могут содержать дуплицированные участки. Поэтому для решения такой задачи необходимо более точно сформулировать условия выравнивания.\\ 
\indent Таким образом, две главные составляющие автоматических методов выравнивания --- это непосредственно алгоритм и функция оценки качества полученного результата. На сегодняшний день можно выделить два основных алгоритма выравнивания биологических последовательностей: алгоритм Нидлмана-Вунша и алгоритм Смита-Ватермана.  Существуют различные их модификации, использующие эвристики для уменьшения количества шагов алгоритма или требуемого объема памяти, однако,эти методы строят выравнивание без проверки биологического смысла результата. В погоне за лучшим счетом происходит потеря качества: множественные разрывы на нуклеотидном и появление стоп-кодонов на аминокислотном уровнях.\\ 
\indent Построение качественного, биологически обоснованного выравнивания нуклеотидных последовательностей с сохранением открытых рамок считывания является очень важной и пока что нерешенной задачей биоинформатики. Для получения парного выравнивания на данный момент существуют программные утилиты,выдающие хороший результат, однако, имеющие столь высокую вычислительную сложность, что не могут быть расширены для построения множественных выравниваний.\\ 
\indent Цель настоящей работы --- разработать и реализовать программу построения множественного выравнивания кодирующих последовательностей ДНК с учетом сдвигов рамки считывания. Для ее достижения нами были поставлены и решены следующие задачи:
\begin{itemize}
\item разработать и реализовать алгоритм парного выравнивания с учетом сдвигов рамки считывания
\item реализовать алгоритм кластеризации для перехода от задачи парного к задаче множественного выравнивания
\item оценить производительность и качество результата созданной программы
\end{itemize}

