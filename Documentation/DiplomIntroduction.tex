\newpage
\part*{\large \centering ВВЕДЕНИЕ}
\addcontentsline{toc}{part}{ВВЕДЕНИЕ}
\hspace{\parindent} Современная биоинформатика --- это молодая, бурно развивающаяся наука, возникшая в 1976-1978 годах и окончательно оформившаяся в 1980 году со специальным выпуском журнала <<Nucleic Acid Research>> (NAR)~\cite{MironovLect}. По сути, это собрание различных математических моделей и методов в помощь биологам для решения биологических задач, таких как: предсказание пространственной структуры белков, расшифровка структуры ДНК, хранение, поиск и аннотация биологической информации.\\
\indent Основу биоинформатики составляют сравнения. Одна из ключевых задач --- поиск сходства последовательностей. Пусть имеются две аминокислотные последовательности и для одной из них известны ее свойства, тогда, если эти последовательности <<похожи>>, можно сделать предположение, что они выполняют сходные функции. Таким образом, новую отсеквенированную последовательность, первым делом, ищут в базах данных известных (аннотированных) последовательностей, чтобы после сравнения судить о том, какие функции она выполняет.\\
\indent Для того, чтобы определить, насколько две последовательности <<похожи>>, используют алгоритмы выравнивания. Они основаны на размещении исходных последовательностей мономеров ДНК, РНК или белков друг под другом таким образом, чтобы легко увидеть их сходные участки~\cite{WikiPairAlign}. Качество выравнивания оценивают, назначая штрафы за несовпадение букв и за наличие пробелов (когда приходится раздвигать одну последовательность для того, чтобы получить наибольшее число совпадающих позиций). При сравнении ищется такой вариант выравнивания, чтобы итоговый счет был максимален.\\ 
\indent Алгоритмы множественного выравнивания, аналогично алгоритмам парного выравнивания, представляют собой инструмент для установления функциональных, структурных или эволюционных взаимосвязей между биологическими последовательностями. Для этой задачи существует <<золотой стандарт>> --- это выравнивание, которое бы получится, если выровнять друг под другом последовательности, имеющие одинаковую пространственную структуру. Это биологически обоснованное выравнивание. Несмотря на то, что задача множественного выравнивания была сформулирована более 20 лет назад~\cite{SIAM_Journal}, она до сих пор не теряет своей актуальности.\\ 
\indent Таким образом, две главные составляющие автоматических методов выравнивания --- это непосредственно алгоритм выравнивания и функция оценки качества полученного результата. На сегодняшний день можно выделить два основных алгоритма выравнивания биологических последовательностей: алгоритм Нидлмана-Вунша и алгоритм Смита-Ватермана. Они представляют собой классический пример задачи динамического программирования. Существуют различные модификации этих алгоритмов, использующие эвристики для уменьшения количества шагов алгоритма или требуемого объема памяти.\\ 
\indent При выравнивании нуклеотидных последовательностей, содержащих открытые рамки считывания, используют косвенные процедуры, которые строят выравнивание исходной последовательности на аминокислотном уровне~\cite{MACSE}. У такого подхода есть несколько проблем. Во-первых, появление преждевременного стоп-кодона. Во-вторых, так как каждая последовательность переводится с одной и той же рамкой считывания от начала и до конца, то присутствие единственного дополнительного нуклеотида приведет к аномальному переводу и выравниванию.