\newpage

\section[Программная реализация]{\large \centering Программная реализация}
\hspace{\parindent} Разработанный алгоритм реализован на языке программирования C++ без использования сторонних библиотек и может быть запущен на всех основных операционных системах: Windows, Linux, Mac OS. Кроме основного приложения, был создан веб-интерфейс, через который также можно получить выравнивание. Так как задача требует существенных вычислительных ресурсов, практично запускать программу на мощной вычислительной платформе, а осуществлять взаимодействие с ней через веб-интерфейс. Кроме этого, для возможности внедрения отдельных компонент программы в другие проекты, были собраны статические библиотеки построения парного и множественного выравниваний.

\subsection[Структуры данных]{\large Структуры данных}
\hspace{\parindent} Ниже описанны используемые в программе структуры для хранения и обработки данных. Они также входят в состав собранных статических библиотек.

\subsubsection[Представление биологических последовательностей]{\large Представление биологических последовательностей}
\hspace{\parindent} 

\subsubsection[Класс построения парного выравнивания]{\large Класс построения парного выравнивания}
\hspace{\parindent} 

\subsubsection[Профили]{\large Профили}
\hspace{\parindent} 

\subsection[Общая схема работы]{\large Общая схема работы}
\hspace{\parindent} Все разбито на отдельные блоки-этапы, блок-схема

\subsubsection[Чтение входных данных]{\large Чтение входных данных}
\hspace{\parindent} Формат входных данных, FASTA и че еще есть
как пиздато одним шмотком память выделяется

\subsubsection[Определение порядка выравнивания]{\large Определение порядка выравнивания}
\hspace{\parindent} Экономия памяти, нет перевыделения строк

\subsubsection[Объединение профилей]{\large Объединение профилей}
\hspace{\parindent} Пиздатый плюсик

\subsection[Параметры выравнивания]{\large Параметры выравнивания}
\hspace{\parindent} графический интерфейс консолька

\subsection[Руководство пользователя]{\large Руководство пользователя}
\hspace{\parindent} параметры варавнивания графический интерфейс консолька